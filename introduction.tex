\documentclass{standalone}
% Load any packages needed for this document
\begin{document}
\starchapter{Введение}
\par Дистанционные курсы, как и дистанционное обучение в целом набирают популярность
в наше время. Существует множество систем для организации дистанционного обучения.
Несмотря на внешние и внутренние различия данные системы функционируют по единому принципу:
приобретение и закрепление знаний. Закрепление знаний чаще всего предполагает проведение
тестирования или выполнение разного рода заданий.
\par Автоматизация тестирования позволяет снизить трудоемкость создания и проведения
дистанционного обучения. Одним из примеров системы управления обучением является Moodle.
Данное СУО позволяет автоматизировать процесс проведения и оценки результатов тестирования.
Более того, за счет открытого исходного кода, система позволяет разрабатывать и адаптировать
способ оценки теста и типы вопросов.
\par Часто в тестах используются вопросы с открытым ответом на естественном языке. Проблема
всех естественных языков в степени их формализованости. Низкая степень формализованности
увеличивает трудоемкость создания и оценки вопроса, так как требует оценки и ввода
всех вариантов эталонного ответа. Более того в ответе так же могут содержаться перечисления,
то есть кортежи элементов, порядок которых не изменяет семантику ответа.
\par Кроме определения уровня подготовки студента, тесты позволяют направить студента.
Подсказки предлагаемые студенту зависят от возможностей системы, но обычно основаны на описании
лексем содержащихся в ответе. Что увеличивает трудоемкость создания вопроса.
\par Данная работа посвящена созданию системы, позволяющей учитывать
последовательности лексем порядок которых не важен.
\newpage
% Your document or picture
\end{document}
