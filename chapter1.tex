\documentclass{standalone}
% Load any packages needed for this document
\begin{document}
\chapter{Анализ современного состояния проблем в области автоматизированной обработки текстов
на естественных языках}%Исследование подходов, методов и средств обработки естественных языков}
\ttl
\section{Существующие подходы к обработке текстов на естественных языках}
\par В наше время обработка естественных языков используется в нескольких направлениях:
\begin{enumerate}
    \item машинный перевод;
    \item информационный поиск;
    \item реферирование текста;
    \item рубрицирование тектса;
    \item обучение языку и др.
\end{enumerate}
\par Несмотря на обилие направлений использования, подходов к обработке всего четыре, символьный, вероятностный, установления связей и гибридный.
\section{Символьный подход}
\par Подход основанный представлении языка как модели сложной, но прозрачной. Примерами такой модели могут послужить:
\begin{enumerate}
    \item обучение на правилах;
    \item индуктивное логическое программирование;
    \item деревья разрешений;
    \item концептуальная кластеризации;
    \item алгоритмы типа k-средних.
\end{enumerate}
\par Более подробно перечисленные методы будут описаны ниже. Общей чертой данных моделей является способ их получение, а именно обучение.
\subsection{Обучение на правилах} % (fold)

\par Один из старейших методов обучение и построения моделей. Используется в случаях когда правила не возможно закодировать правила иначе, то есть когда правила являются эвристическими.

\subsection{Индуктивное логическое программирование} % (fold)

\par Это раздел машинного обучения использующий в качестве примеров, фоновых знаний и гипотез логическое программирование. Логическое программирование — это парадигма программирования, которая основана на автоматическом доказательстве теорем. Логическое программирование основано на теории и аппарате математической логики с использованием математических принципов резолюций.

\subsection{Деревья разрешений} % (fold)

\par Это средство принятия решений, используемое в прогнозировании и обработке данных. Структура дерева представляет собой «листья» и «ветки». На ребрах («ветках») дерева решения записаны атрибуты, от которых зависит целевая функция, в «листьях» записаны значения целевой функции, а в остальных узлах — атрибуты, по которым различаются случаи. Для определения значения, необходимо спуститься по дереву до листа и вернуть его значение.

\subsection{Концептуальная кластеризация} % (fold)

\par Кластеризация является еще одним способом обработки естественных языков. Кластеризация является задачей обучения без учителя. Суть метода заключается в обучении на большой выборке, позволяющий выделить объекты в однородные группы. Ниже приведена классификация методов кластеризации являющаяся общепринятой:
\begin{enumerate}
    \item вероятностный подход:
        \begin{enumerate}
            \item метод К-средних;
            \item метод К-medians;
            \item EM-алгоритм;
            \item алгоритм семейства FOREL;
            \item дискриминантный анализ,
        \end{enumerate}
    \item методы на основе систем искусственного интеллекта:
        \begin{enumerate}
            \item метод нечеткой кластеризации С-средних;
            \item нейронная сеть Кохонена;
            \item генетический алгоритм,
        \end{enumerate}
    \item логический подход. Кластеризация на основе дерева решений,
    \item теоретико-графический подход:
        \begin{enumerate}
            \item графические алгоритмы кластеризации;
        \end{enumerate}
    \item иерархический подход. Используется в ситуации наличия подгрупп внутри групп:
        \begin{enumerate}
            \item агломеративные алгоритмы;
            \item дивизивные алгоритмы,
        \end{enumerate}
    \item остальные методы:
        \begin{enumerate}
            \item статистические методы кластеризации;
            \item ансамбль кластеров;
            \item алгоритмы семейства KRAB;
            \item алгоритм, основанный на методе просеивания DBSCAN и др.
        \end{enumerate}
\end{enumerate}

\subsection{Алгоритмы типа k-средних} % (fold)

\par Наиболее популярный алгоритм кластеризации, стремящийся минизировать суммарное квадратичное отклонение точек кластеров от центров этих кластеров.
\par Основная идея заключается в том, что на каждой итерации вычисляется центр масс для каждого кластера, полученного на предыдущем шаге, затем векторы разбиваются на кластеры вновь в соответствии с тем, какой из новых центров оказался ближе по выбранной метрике.
\par Алгоритм завершается, когда на какой-то итерации не происходит изменения центра масс кластеров. Это происходит за конечное число итераций, так как количество возможных разбиений конечного множества конечно, а на каждом шаге суммарное квадратичное отклонение не увеличивается, поэтому зацикливание невозможно.

% subsection  (end)
% subsection  (end)
% subsection  (end)
% subsection  (end)
% subsection  (end)
% subsection  (end)
\section{Вероятностный подход}
\par Подход использует различные математические техники, а также большие текстовые корпуса для разработки обобщенных моделей языковых явлений, базой для которой является реальные примеры найденные в текстовом корпусе не используя дополнительных знаний о языке или о внешнем мире. Основное отличие от символьного подхода использование реальных данных в качестве первичного источника информации.
\par В вероятностном подходе существует несколько течений, среди которых особого внимания заслуживают модели, максимизирующие энтропию и скрытые марковские модели (СММ). СММ  это конечный автомат, который имеющий множество состояний с определенными вероятностями переходов между ними. Каждое состояние производит один из наблюдаемых результатов с определенной вероятностью. Хотя результаты являются видимыми, но состояние модели скрыто от внешнего наблюдения. Главным преимуществом вероятностных моделей заключено в том, что они дают способ решения многих видов неоднозначных проблем, формулируемых так "с учетом N некоторых неоднозначных вводов выбрать один наиболее вероятный".

\subsection{Методы максимизации энтропии} % (fold)

\par Данный метод классификации основан на понятии информационной энтропии. Информационная энтропия - это мера неопределенности вероятностного распределения.
\par \textbf{Еще не нашел хорошего описания этого алгоритма.}

\subsection{Скрытые макрковские модели} % (fold)
\par Скрытая марковская модель — статистическая модель, имитирующая работу процесса, похожего на марковский процесс с неизвестными параметрами, и задачей является определение неизвестных параметров на основе наблюдаемых. Полученные параметры могут быть использованы в дальнейшем анализе.

% subsection  (end)
% subsection  (end)
\section{Подход установления связей}
\par Подход установление связей основан на моделях массивных связанных наборов простых и нелинейных компонентов. Эти компоненты работают параллельно. Приобретенное в результате обработки знание сохраняется в образце весов взаимосвязи компонентов.
\section{Гибридный подход}
\par Гибридные методы используют преимущества трех только что описанных подходов, минимизируя человеческие усилия, требуемые для типовой лингвистической конструкции и максимизируя гибкость, эффективность, и надежность применения NLP при человеко-компьютерном взаимодействии.

\par При всех подходах обработка языка, как правило, включает элементы машинного обучения: модель классификации и обучающую последовательность. На основании описания атрибутов каждого объекта модель классификации относит каждый объект в какой-то класс, обучающая последовательность ставит в соответствие последовательности объектов последовательность классов.

\section{Выбор программного средства синтаксического анализа}

\par Синтаксический анализ является сложным и трудоемким процессом, в особенности если речь идет об анализе естественного языка. Именно это стало причиной принятия решения о использовании стороннего средства синтаксического анализа.
\par Было рассмотрено множество синтаксических анализаторов и выделены критерии сравнения. Наиболее развитыми на момент написания диссертации являются следующие анализаторы:
\begin{enumerate}
    \item анализатор Стенфорского университета;
    \item анализатор Института Брауна;
    \item анализатор Института Беркли;
    \item анализатор Института Токио.
\end{enumerate}
\subsection{Критерии сравнения}
\par От выбора критериев сравнения зависит решения о выборе средства синтаксического анализа, что в конечном итоге повлияет на результат все проделанной работы.
\par Moodle является системой управления обучением с открытым исходным кодом, что накладывает ограничения на используемое с ним программное обеспечение. Поэтому лицензия средства синтаксического анализа является важным критерием отбора. Каждый из рассматриваемых средств синтаксического анализа имеет открытую лицензию, которая позволяет использовать это средство в связке с Moodle.
\par Следующим важным критерием отбора является наличие тестов для выбираемого программного средства. Так как наличие большого количества тестов и покрытие этими тестами кода, указывает на качество программного средства, а так же позволяет оценить его функциональные возможности.
\par Определение части речи, к которой принадлежит лексема, является минимальной необходимой функциональностью для использования для определения перечислений в тексте написанном на естественном языке. Наличие этого критерия является обязательный требованием к программному средству синтаксического анализа.
\par Еще одним критерием сравнения является язык на котором написано программное средство. Этот критерий повлияет на системные требования предъявляемые разрабатываемым программным средством.
\par Одним из важных критериев является способ взаимодействия разрабатываемого программного средства со средством синтаксического анализа.
\par В качестве дополнительного критерия выступает возможность определение средством синтаксического анализа связей между лексемами в предложении.

\subsection{Синтаксический анализатор Института Брауна}

\par Данный синтаксический анализатор разрабатывается начиная с 2000 года. И сильно изменился с первой версии. Это синтаксический анализатор основанный на методе максимизации энтропии и модели самообучения. Данное программное средство разработано на языке Python и предполагает использование по средствам вызова из командной строки. К исходному коду парсера прилагается набор тестов, позволяющих оценить качество написанного программного средства.
\par Возможности данного пасера ограничиваются определением членов предложения и принадлежащих им лексем.

\subsection{Синтаксический анализатор Института Беркли}

\par Данное программное средство разрабатывается Институтом Беркли с 2001 года. В основу парсера лег символьный подход к синтаксическому анализу, алгоритм К-best. Данное программное средство написано на языках Java и Scala. И так же как рассмотренное ранее средство синтаксического анализа предполагает запуск с аргументами из командной строки.
\par Возможности данного парсера аналогичны возможностям ранее рассмотренного программного средства.

\subsection{Синтаксический анализатор Института Токио}

\par Институт Токио занимается разработкой средства синтаксического анализа английского языка с 2005 года. Программное средство использует подход установления связей в процессе анализа текста, алгоритм построения структуры предложения управляемой главным членом предложения. Язык написания данного программного средства C++. В отличие от двух предыдущих средств синтаксического анализа данное помимо запуска из командной строки предлагает возможность запуска в качестве сервера, с которым возможно общение по протоколу HTTP.
\par Возможности данного средства синтаксического анализа расширяют возможности средств описанных выше, за счет определения связей между членами предложения.

\subsection{Синтаксический анализатор Стэнфордского университета}

\par Данное средство синтаксического анализа естественных языков ведет свою историю с 1990 года. Этот синтаксический анализатор является самым развитым на данный момент. Данное программное средство использует гибридный подход к синтаксическому анализу. В качестве языка написания данного программного средства выступает язык Java. Тестовая база данного синтаксического анализа включается в себя более 10000 тестов. Так же как и предыдущий анализатор данный предлагает два режима работы, в качестве средства командной строки, а так же в качестве отдельного сервера, с возможность выполнения запроса к нему по протоколу HTTP.
\par Возможности данного программного средства синтаксического анализа аналогичны возможностям предыдущего средства. С одной лишь разницей, данное средство имеет более высокие показатели качества за счет использования более продвинутого подхода к синтаксическому анализу.

\subsection{Выводы}

\par В результате изучения представленных выше программных средств синтаксического анализа, было принято решение выбрать для использование программное обеспечение разработанное на базе университета Стенфорда. Причинами для такого решения послужили следующие аргументы:
\begin{enumerate}
    \item синтаксический анализатор Стенфорского университета использует обобщенное описание синтаксиса языка, что позволяет использовать его с другими естественными языками;
    \item это наиболее активно развивающийся синтаксический анализатор;
    \item данный анализатор имеет наибольшую тестовую базу, что позволяет удостовериться в качестве его работы на наибольшем количестве тестовых ситуаций.
\end{enumerate}

\end{document}
