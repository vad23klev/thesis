\documentclass{standalone}
% Load any packages needed for this document
\begin{document}
% Your document or picture
\chapter{Выделение структур английского языка удовлетворяющих перечислениям}%Исследование грамматики английского языка}
\ttl
\section{Определение структур языка являющихся перечислениями}
\par Определение структур языка являющийся перечислениями сводится определению структур языка, перестановка элементов которых, не изменяет семантического смысла предложения и не нарушают синтакического строения предложения. Примерами таких структур служат однородные члены предложения:
\begin{enumerate}
    \item сказуемые;
    \item подлежащие;
    \item определения;
    \item дополнения;
    \item обстоятельства.
\end{enumerate}
\par Другим примером служит сложносочиненные преложения.
\par Рассмотрение каждой структуры в отдельности позволит построить модель предложения на естественном языке, и построить модель определения перечислений для естественного языка.
\subsection{Сказуемое} % (fold)
We went to the cafe and buy a cup of coffee.
\subsection{Подлежащее} % (fold)
Mary and Jane keep their chidren in secret.
\subsection{Определение} % (fold)
The weather today is cloudy, rainy, foggy and freezing-cold.
\subsection{Дополнение} % (fold)
We need your name, photo and number.
\subsection{Обстоятельства} % (fold)
I go to the cinema, cafe and to the park.
\subsection{Сложносочиненые предложения} % (fold)
I see a big group of people which contains three subgroups: children, women and men, who were dressed in blue overalls or strange suits with red and green lines.

\section{Построение модели}
\end{document}
